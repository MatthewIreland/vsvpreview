% Example of VSV document.
% Matthew Ireland June/July MMXV

\mag=1200

%%% VSV IN-VIDEO TEMPLATE
%%% MATTHEW IRELAND, 23 JULY MMXV

\documentclass[10pt]{article}

%%% VSV IN-VIDEO TEMPLATE
%%% MATTHEW IRELAND, 23 JULY MMXV

% PACKAGES USED IN VSV TEMPLATE
% FEEL FREE TO ADD TO THIS LIST

\usepackage[utf8]{inputenc}
\usepackage{amsmath}
\usepackage{amssymb}
\usepackage{hyperref}
\usepackage{graphicx}
\usepackage{fancyhdr}
\usepackage{color,soul}
\usepackage[table,usenames,dvipsnames]{xcolor} % table cell colouring
\usepackage{array,multirow,graphicx}
\usepackage{longtable}     % split large tables over page boundaries
\usepackage{hhline}
\usepackage{enumerate}
\usepackage{hyphenat}
\usepackage{fancyvrb}
\usepackage{epstopdf}
\usepackage{semantic} % for inference rules
\usepackage[utf8]{inputenc}
\usepackage{changepage}   % adjustwidth environment
\usepackage{cleveref} % referencing with § character
\usepackage{framed,color}
\usepackage{makecell}
\usepackage[font=small,labelfont=bf]{caption} % small captions, bold labels
\usepackage{lipsum}
\usepackage{listings}      % source-code listings
\usepackage{todonotes}     % todo notes (inc. placeholder images)
\usepackage{siunitx}       % si units
\usepackage{float} % custom floats and for 'H' placement specifier on figures
\usepackage{complexity}    % complexity classes typesetting
\usepackage{pdftexcmds}    % conditionals (for setting text width)
\usepackage{mdframed}


% tikz
\usepackage{pgf}
\usepackage{tikz}
\usetikzlibrary{arrows,shapes,trees,backgrounds,automata,patterns}


%%%% PDFINFO FOR PDFTEX %%%%
\pdfinfo{/Author (jkf21 or mti20)
         /Title (VSV)
         /Keywords (vsv)}
%%%% END PDFINFO %%%%


%% Common layout elements
%% Matthew Ireland, 23 July MMXV

% Font definitions
% Upgraded from those in FHK's FORMAT.tex to use
% Latin Modern for vectorization & additional glyphs.
\usepackage{lmodern}
\usepackage[T1]{fontenc}


% Length definitions
\makeatletter
\g@addto@macro\normalsize{%
 %% page layout
   % see corresponding documents for left/right frames
 %% paragraphs
   \setlength{\baselineskip}{12pt plus 0pt minus 0pt}
   \setlength{\parskip}{12pt plus 0pt minus 0pt}
   \setlength{\parindent}{0pt plus 0pt minus 0pt}
 %% floats
   \setlength{\floatsep}{12pt plus 0 pt minus 0pt}
   \setlength{\textfloatsep}{20pt plus 0pt minus 0pt}
   \setlength{\intextsep}{14pt plus 0pt minus 0pt}
   \setlength{\dbltextfloatsep}{20pt plus 0pt minus 0pt}
   \setlength{\dblfloatsep}{14pt plus 0pt minus 0pt}
 %% maths
   \setlength{\abovedisplayskip}{12pt plus 0pt minus 0pt}
   \setlength{\belowdisplayskip}{12pt plus 0pt minus 0pt}
   \setlength{\abovedisplayshortskip}{6pt}
   \setlength{\belowdisplayshortskip}{6pt}
 %% lists
   \setlength{\topsep}{10pt plus 0pt minus 0pt}
   \setlength{\partopsep}{3pt plus 0pt minus 0pt}
   \setlength{\itemsep}{5pt plus 0pt minus 0pt}
   \setlength{\labelsep}{8mm plus 0mm minus 0mm}
   \setlength{\parsep}{\the\parskip}
   \setlength{\listparindent}{\the\parindent}
 %% verbatim
   \setlength{\fboxsep}{5pt plus 0pt minus 0pt}
}
\makeatother


% PERMIT GAPPY TEXT IN PREFERENCE TO HYPHENS
\hyphenpenalty=10000
\tolerance=10000


% make links not-ugly
\hypersetup{
    colorlinks=false,
    pdfborder={0 0 0},
}

%%%% referencing with § character %%%%
\crefformat{section}{§#2#1#3}
\crefname{section}{§}{§§}
\Crefname{section}{§}{§§}

% frames
\definecolor{shadecolor}{rgb}{1,0.8,0.3}


% tables -- centre text vertically
\newcolumntype{L}[1]{>{\raggedright\let\newline\\\arraybackslash\hspace{0pt}}m{#1}}
\newcolumntype{C}[1]{>{\centering\let\newline\\\arraybackslash\hspace{0pt}}m{#1}}
\newcolumntype{R}[1]{>{\raggedleft\let\newline\\\arraybackslash\hspace{0pt}}m{#1}}


%%% VSV macro definitions


% FHK's \marks definition, repackaged as LaTeX
\newcommand{\vsvmarks}[1]{{\unskip \nobreak \hfil \penalty 50 \hskip 8pt
                \hbox{}\nobreak \hfil \setbox0=\hbox{#1}%
                [\ifdim \wd0>11pt #1\else #1 mark\ifnum #1=1\else s\fi\fi]%
                \parfillskip=0pt \finalhyphendemerits=0\par}}

% FHK's \def macro, to be repackaged as LaTeX (TODO)
\def \Def {\buildrel {\rm def} \over = }

% TDJ's \examhead
\newcommand{\vsvexamhead}[3]{\section{#1 Paper #2 Question #3}}

% This is an attempt to implement FHK's \beginquestion in LaTeX
% Usage: \begin{vsvquestion}{<year>}{<paper>}{<question num>}
\newenvironment{vsvoldquestion}[3]%
 {\subsection*{#1 Paper #2}\begin{minipage}[t]{24pt}{\textbf{#3}}\end{minipage}\begin{minipage}{\textwidth-24pt}}
 {\end{minipage}\vspace{12pt}}


% This is a better way to do questions
\newenvironment{vsvquestion}[3]%
 {\subsection*{#1 Paper #2 Question #3}}
 {\vspace{12pt}}


\newcommand{\vsvitem}[3]{\item}

\definecolor{hlcolour}{rgb}{1,1,0}
\sethlcolor{hlcolour}
%\newcommand{\vsvhl}[1]{\hspace{-2pt}\hl{\mbox{}~#1~\mbox{}}\hspace{-2pt}}
\newcommand{\vsvhl}[3]{\hl{#3}}

\newcommand{\vsvhlnowrap}[3]{\colorbox{hlcolour}{{#3}}}

\setstcolor{red}
\newcommand{\vsvcorrect}[4]{\st{#3}\ \ \ \ \ #4}
\newcommand{\vsvxcorrect}[2]{\vsvcorrect{}{}{#1}{#2}}
\newcommand{\vsvxxcorrect}[2]{#1}

\lstdefinestyle{customc}{
  belowcaptionskip=1\baselineskip,
  breaklines=true,
  xleftmargin=\parindent,
  language=C,
  showstringspaces=false,
  basicstyle=\fontsize{11pt}{13pt}\selectfont\ttfamily,
  columns=flexible,
  keepspaces=true,
  keywordstyle=\bfseries\color{green!40!black},
  commentstyle=\itshape\color{purple!40!black},
  identifierstyle=\color{blue},
  stringstyle=\color{orange},
  escapechar=~
}




\newenvironment{vsvgrey}[2]{\leavevmode\color{gray}}{\leavevmode\color{black}}

\newenvironment{vsvappear}[2]{}{}

\newcommand{\nosection}[1]{%
  \refstepcounter{section}%
  \addcontentsline{toc}{section}{\protect\numberline{\thesection}#1}%
  \markright{#1}}


%%%% preview environment %%%%
\newcounter{vsvrhscounter}
\newcounter{vsvlhscounter}
\makeatletter
\newcommand\vsvsetwidth[3]{%
  \ifnum\pdf@strcmp{\unexpanded{#1}}{right}=0 %
     \expandafter\@firstoftwo
  \else
    \expandafter\@secondoftwo
  \fi
    {\eject \pdfpageheight=171mm \newgeometry{textheight=143mm,textwidth=129mm,top=28mm,bottom=10mm,left=14mm,showframe=true}\fancyhf[LC]{}\fancyhf[HC]{\textbf{Question (right-hand) pane -- Number R\thevsvrhscounter}\\In: #2 --- Out: #3}\stepcounter{vsvrhscounter}}
    {\eject \pdfpageheight=105mm \newgeometry{textheight=67mm,paperwidth=149mm,textwidth=129mm,top=28mm,bottom=10mm,left=14mm,showframe=true}\fancyhf[LC]{}\fancyhf[HC]{\textbf{Image/code (left-hand) pane -- Number L\thevsvlhscounter}\\In: #2 --- Out: #3}\stepcounter{vsvlhscounter}}%
}
\makeatother



\newenvironment{vsvframe}[3]{%
\vsvsetwidth{#3}{#1}{#2}
\thispagestyle{fancy}

\begin{mdframed}
}%
{\end{mdframed}

\eject \pdfpageheight=297mm
\restoregeometry
\pagestyle{empty}}


\usepackage{geometry}
\usepackage{a4}

\begin{document}
\raggedbottom


\section*{Examples}

This provides a very brief overview of the core functionality of the VSV preview script.
Also see the \verb=test_cases= folder for a wider range of examples.

\subsection*{Greying out text}
\begin{vsvgrey}{}{1-00:20}
Grey out blocks of text like this.
\end{vsvgrey}

\begin{itemize}
\vsvitem{}{1-00:30}{grey}
In lists, use the \verb=\vsvitem= macro in lists instead of \verb=\item=.
\end{itemize}

The first two arguments to \verb=\vsvitem= follow the standard timecode convention (described in the Word document).
Leaving the third argument empty reduces \verb=\vsvitem= to be exactly the same as \verb=\item=.
Alternatively, specify \verb=grey= or \verb=appear= in the third argument to indicate whether the item should be un-greyed or should appear afresh at the timecode in the first argument.
\verb=\vsvitem= can also be used in \verb=enumerate= and \verb=description= lists, as if it were \verb=\item=.

\subsection*{Making text appear}
\begin{vsvappear}{1-00:40}{}
Make blocks of text appear like this.
\end{vsvappear}

\begin{itemize}
\vsvitem{1-00:50}{}{appear}
In lists, again use the \verb=\vsvitem= syntax.
\end{itemize}

\subsection*{Highlighting}
Text \vsvhl{2-00:00}{2-00:10}{can be highlighted} like this.
If the lines are long, this method will \vsvhl{27-15}{}{wrap around} the line ends.

%Alternatively, \vsvhlnowrap{40-04}{}{highlight like this} to avoid the shrinkwrap appearance.
%However, such highlights cannot be wrapped around line ends (yet -- working on it).


\subsection*{Strikethrough corrections}
A correction can be placed to the right of standalone text like this:

This is the incorrect text.

\vsvcorrect{27-15}{}{This is the incorrect text.}{This is the corrected text}

\subsection*{Code}
Use the Listings package for automated syntax highlighting as follows:

\lstset{style=customc}
\begin{lstlisting}
 tmp1 = a*b;
 x = tmp1+c;    // this is a comment
\end{lstlisting}

Corrections can be made as follows:

\lstset{style=customc}
\begin{lstlisting}
~ \vsvcorrect{27-15}{}{tmp1=a*b}{tmp1=a-b}~
 x = tmp1+c;    // this is a comment
\end{lstlisting}

\subsection*{Marks}

FHK's \verb=\marks= macro has been imported as \verb=\vsvmarks=\vsvmarks{5}.

\subsection*{Frame Environments}

Examples of frames (1-per-page):

\begin{vsvframe}{1-00:00}{2-10:00}{right}
\begin{enumerate}[(a)]
\vsvitem{1-00:10}{}{appear}
\lipsum[1]
\vsvitem{1-00:20}{}{appear}
\lipsum[2]
\vsvitem{2-00:00}{}{appear}
\lipsum[68]
\vsvitem{2-00:10}{}{appear}
Filling to the end... more text... more text... more text... more text... more text... more text... more text... more text... more text... more text... more text... more text... more text... more text... more text...
\end{enumerate}
\end{vsvframe}

\begin{vsvframe}{1-00:00.10}{1-00:10.20}{left}
\begin{enumerate}[(a)]
\vsvitem{}{1-00:010.20}{grey}
\lipsum[1]
\end{enumerate}
\end{vsvframe}

\input{meta/vsvftr.tex}
